\documentclass[draft]{phd}

\begin{document}
	%
	\logotypes{./Resources/Logos/LPTHE.pdf}{./Resources/Logos/ILPlogo.pdf}{./Resources/Logos/UPMC.pdf}
	%\logotypes{./Resources/Logos/UPMC.pdf}{}{./Resources/Logos/LPTHE.pdf}
	%
	\speciality{Physique Théorique}
	\doctoralschool[https://www.edpif.org/en/index.php]{Physique en île-de-France}
	\university[http://www.upmc.fr/]{Pierre et Marie \textsc{Curie}}
	\laboratory[http://www.lpthe.jussieu.fr/]{Laboratoire de Physique Théorique et Hautes Énergies}
	%
	\name{Oscar}{\href{mailto:odefelice@lpthe.jussieu.fr}{de Felice}}
	%
	\title[Solutions avec flux et Géométrie Généralisée Exceptionnelle]{Flux backgrounds and Exceptional Generalised Geometry}
	%
	\abstract[%
			Cette thèse traite de compactifications avec flux en théorie des cordes et supergravité.\\
			D'abord, nous étudions les reductions dimensionnelles des théories de type II et de supergravité en onze dimensions, en utilisant la géométrie généralisée exceptionnelle.\\
			Nous commençons par l'introduction des techniques mathématiques necessaire à cette thèse, nous nous concentrons sur les $G$-structures et leur extension à la géométrie généralisée.\\
			Après, nous passons à discuter les compactifications à proprement parler.
			Précisément, nous nous concentrons sur type IIA, en construisant la version de la géométrie généralisée exceptionnelle décrivant cette supergravité et en trouvant les deformations de la dérivé de Lie généralisée correctes qui permettre de tenir compte et décrire correctement la mass de Romans.
			Nous présentons la méthode de Scherk-Schwarz généralisée qui nous permettre de trouver des ansatze consistants qui préservent la quantité maximale de supersymétrie.
			Aussi, nous appliquons cette méthode à des examples différents des truncations sur les spheres, nous sommes capables de reproduire l'ansatz sur la sphere six-dimensionnelle et le tensor d'imbrication, qui nous donne une supergravité jaugée $\ISO(7)$ dyoniquement  en quatre dimensions.
			Pour des spheres de dimension $d=2,3,4$, nous trouvons une obstruction à avoir des parallelisations généralisées dans les cas massifs. 
			Ceci donne une indication du fait que des reductions dimensionnelles en presence de mass de Romans peut pas exister.\\
			En outre, nous étudions les calibrations générales sur des backgrounds AdS en type IIB et M-théorie.
			Nous établissons que elles sont décrites par les structures de Sasaki-Einstein exceptionnelles, et nous focalisons notre attention sur les vectors de Reeb généralisés.
			Les inégalités pour la limite sur l'énergie peuvent être dérivées par la decomposition de la condition donnée par la symétrie $\kappa$ ou dans la même façon, par la decomposition des bilinéaires des champs spinoriels existants en literature.
			Nous expliquons comme la fermeture des formes de calibration est liée à l'integrabilité de la structure de Sasaki-Einstein exceptionnelle décrivant le background.
			En particulier, nous faisons ça pour des branes remplissants l'espace ou ponctuelles.
			En faisant ça, nous montrons que la partie de forme du vector twisté en M-théorie donne les correctes calibrations généralisées.
			Le cas au sujet des background en type IIB donne des résultats analogues.%
		]{%
					The main topic of this thesis are flux compactifications.\\
					Firstly, we study dimensional reductions of type II and eleven-dimensional supergravities using exceptional generalised geometry.\\
					We start by presenting the needed mathematical tools, focusing on $G$-structures and their extension to generalised geometry.\\
					Then, we move our focus on compactifications. 
					In particular, we mainly focus on type IIA, building the version of exceptional generalised geometry adapted to such supergravity and finding the right deformations of generalised Lie derivative to accomodate the Romans mass.
					We describe the generalised Scherk-Schwarz method to find consistent truncation ansatze preserving the maximal amount of supersymmetry.
					We apply such a method to several examples of truncations on spheres, we reproduce the truncation ansatz on $S^6$ and the embedding tensor leading to dyonically gauged $\ISO(7)$ supergravity in four dimensions.
					For spheres of dimension $d=2,3,4$, we find an obstruction to have generalised parallelisations in massive theory, giving the evidence that maximally supersymmetric reductions might not exist.\\			
					As further point, we study generalised calibrations on AdS backgrounds in type IIB and M-theory. 
					We find these are described by Exceptional Sasaki-Einstein structures and we place the focus on the generalised Reeb vectors. 
					The inequalities for the energy bound are derived by decomposing a $\kappa$-symmetry condition and equivalently, bispinors in calibration conditions from existing literature. 
					We explain how the closure of the calibration forms is related to the integrability conditions of the Exceptional Sasaki-Einstein structure, in particular for AdS space-filling or point-like branes. 
					Doing so, we show that the form parts of the twisted vector structure in M-theory provides the expected generalised calibrations. 
					The IIB case yields similar results.
					 %
				}
	%
	\defensedate{26 Mars 2018}
	%
	\addjurymember{M}{Henning}{\href{mailto:henning.samtleben@ens-lyon.fr}{Samtleben}}{Rapporteur}
	\addjurymember{M}{Alberto}{\href{mailto:Alberto.Zaffaroni@mib.infn.it}{Zaffaroni}}{Rapporteur}
	\addjurymember{F}{Anna}{\href{mailto:ceresole@to.infn.it}{Ceresole}}{Examinateur}
	\addjurymember{F}{Mariana}{\href{mailto:mariana.grana@cea.fr}{Gra\~na}}{Examinateur}
	\addjurymember{M}{Jan}{\href{mailto:jan.troost@lpt.ens.fr}{Troost}}{Examinateur}
	\addjurymember{F}{Michela}{\href{mailto:petrini@lpthe.jussieu.fr}{Petrini}}{Directrice de thèse}
	%
	\titlepage
	%
	\coverpage[Arthur \textsc{Rimbaud}]{Un soir, j'ai assis la Beauté sur mes genoux. -- Et je l'ai trouvée amère -- Et je l'ai injuriée.}
	%
	\dedicationpage[To Martino,\\ for the hope. \\ To Gabri and Betta, \\ for the bravery.]
	%
\end{document}