\documentclass[debug]{phd}

\begin{document}
	%
	\chapter{Notations and Conventions}
	\label{app:notation}
		%
			%
			The indices used in this thesis -- if not differently indicated -- are:
				%
					\begin{align*}
						\mu, \nu  \ &: \ \text{external spacetime indices} \, ,\\
						m,n \ &:\ \text{curved indices on the internal manifold $M_d$} \, ,\\
						a,b \ &:\ \text{frame indices on $M_d$} \, ,\\
						i,j \ &:\ \text{indices for the embedding coordinates of $S^d$ in $\mathbb R^{d+1}$ (or $H^{p,q}$ in $\RR^{p+q}$)} \, ,\\
						I,J \ &:\ \text{$\SL(d+2,\RR)$ indices}\, , \\
						M,N \ &: \ \text{curved indices for the $E_{d+1(d+1)}\times \RR^+$ generalised tangent space on $M_d$} \, ,\\
						A,B \ &: \ \text{frame indices for the $E_{d+1(d+1)}\times \RR^+$ generalised tangent space on $M_d$} \, .
					\end{align*}
				%

			Our tensor conventions are the same as in~\cite{waldram2}. 
			We collect here the ones relevant for our computations. 
			On a $d$-dimensional manifold $M_d$, given a form $\lambda \in \Lambda^p T^*$ and a poly-vector $w \in \Lambda^q T$, 
				%
					\begin{equation}
						\lambda = \frac{1}{p!} \lambda_{m_1\ldots m_p} \dd x^{m_1} \wedge \cdots \wedge \dd x^{m_p}\ ,\qquad w = \frac{1}{q!} w^{m_1 \ldots m_q} \frac{\partial}{\partial x^{m_1}}\wedge \cdots \wedge\frac{\partial}{\partial x^{m_q}} \,,
					\end{equation}
				%
			we define the contraction
				%
					\begin{equation}\label{deflrcorner}
						\begin{array}{l r}
							(w \,\lrcorner\,\lambda)_{m_1\ldots m_{p-q}} = \frac{1}{q!}w^{n_1 \ldots n_q}\lambda_{n_1\ldots n_q m_1 \ldots m_{p-q}} &\mbox{if}\ q \leq p \, , \\[1mm]
							(w \,\lrcorner\,\lambda)^{m_1\ldots m_{q-p}} = \frac{1}{p!}w^{m_1 \ldots m_{q-p}n_1 \ldots n_p}\lambda_{n_1\ldots n_p} &\mbox{if}\ p< q\, .
						\end{array}
					\end{equation}
				%
			The contraction of a vector $v\in T$ with a form $\lambda$ is also denoted by $\iota_v \lambda \equiv v\,\lrcorner \,\lambda$.

			The contraction of a poly-vector $w$ with a tensor $\tau \in T^*\otimes \Lambda^d T^*$ is defined as
				%
					\begin{equation}
						(w\,\lrcorner\, \tau)_{m_1\ldots m_{d-q+1}} = \frac{1}{(q-1)!} w^{n_1\ldots n_q} \tau_{n_1,\,n_2\ldots n_q m_1\ldots m_{d-q+1}} \,.
					\end{equation}
				%

			Moreover, for $\lambda \in \Lambda^{p}T^*$ and $\mu \in \Lambda^{d-p+1}T^*$, we define the \emph{$j$-operator} giving  $j \lambda \wedge \mu \in T^* \otimes \Lambda^d T^*$ as: 
				%
					\begin{equation}\label{joper}
						\left(j \lambda \wedge \mu\right)_{m,\,m_1\ldots m_d} = \frac{d!}{(p-1)!(d-p+1)!}\,\lambda_{m[m_1\ldots m_{p-1}}\mu_{m_p\ldots m_d]}\ .
					\end{equation} 
				%
			This is the same as $j\lambda \wedge \mu = \dd x^m \otimes (\iota_m \lambda \wedge \mu)$. Upon exchanging $\lambda$ and $\mu$ one has
				%
					\begin{equation} 
						j \lambda \wedge \mu = (-1)^{p(d-p+1)+1}\, j\mu \wedge \lambda\ .
					\end{equation}
				%

			For the Hodge star we take
				%
					\begin{equation}
						(*\lambda)_{m_1\cdots m_{d-p}} = \frac{1}{p!}\sqrt{g}\,\epsilon_{m_1\cdots m_{d-p}}^{\phantom{m_1\cdots m_{d-p}} n_1\ldots n_p}\lambda_{n_1\ldots n_p} \ ,
					\end{equation}
				%
			with $\epsilon_{1\ldots d} = +1$.

			The action of a $\mathfrak{gl}(d)$ element $r \in T \otimes T^*$ on a vector $v\in T $ and on a $p$-form is defined as
				%
					\begin{equation}
						(r \cdot v)^m = r^{m}_{\phantom{m}n} v^n\,, \qquad (r\cdot \lambda)_{m_1\ldots m_p} = -p\, r^{n}_{\phantom{n}[m_1} \lambda_{|n| m_2\ldots m_p]}  \,. 
					\end{equation}
				%
			%
			\section{Constrained coordinates on the spheres}\label{app:constrcoo}
				%
				In the following we provide some useful formulae for the embedding coordinate description of the round sphere $S^d$, mostly taken from~\cite{spheres}. 
				These are needed to study the parallelisations of the exceptional tangent bundle presented in the main text.

				We parameterise $\RR^{d+1}$ in Cartesian coordinates as $x^i = r\, y^i$, $i=1,\ldots d+1$, with 
				%
					\begin{equation}
						\delta_{ij}\,y^iy^j \,=\, 1\ .
					\end{equation}
				% 
				Then the $d$-dimensional sphere $S^d$ of radius $R$ is obtained by fixing $r=R$. 
				The standard metric and volume form on $\RR^{d+1}$ induce the following round metric and \hbox{volume form on $S^d$:}
				%
					\begin{equation}\label{roundSdmetric}
						\rg{g} = R^2 \, \delta_{ij}\dd y^i \dd y^j \ ,
					\end{equation}
				%
					\begin{equation}
						\rg{\rm vol}_{d} \, = \, \frac{R^d}{d!} \,\epsilon_{i_1 \ldots i_{d+1}} y^{i_1} \dd y^{i_2}\wedge \cdots \wedge \dd y^{i_{d+1}}\ .
					\end{equation}
				%
				The Killing vector fields generating the $SO(n+1)$ isometries can be written as 
					%
						\begin{equation}
							v_{ij} \,=\, R^{-1}\left(y_i k_j - y_j k_i \right)\ ,
						\end{equation}
					%
				where $k_i$ are conformal Killing vectors, satisfying 
					%
						\begin{align}
							\mathcal{L}_{k_i}\! \rg{g} &= - 2 y^i \rg{g}\ , \\
										k_i (y_j) &:= \iota_{k_i}\dd y_j \, = \, \delta_{ij} - y_i y_j\ .
						\end{align}
					%
				The index on the coordinates $y^i$ is lowered using the $\RR^{d+1}$ metric $\delta_{ij}$.
				The Killing vectors $v_{ij}$ generate the $\mathfrak{so}(n+1)$ algebra,
					%
						\begin{equation}\label{algebra_Killing_v}
							\mathcal{L}_{ v_{ij}} v_{kl} \, =\, R^{-1}\left(\delta_{ik}v_{lj} - \delta_{il}v_{kj} - \delta_{jk}v_{li} + \delta_{jl} v_{ki} \right)\ ,
						\end{equation}
					%
				while the constrained coordinates $y_k$ and their differentials $\dd y_k$ transform in the fundamental representation of $SO(n+1)$ under the Lie derivative, 
					%
						\begin{equation}\label{Lie_on_y}
						\begin{split}
							\mathcal{L}_{v_{ij}}y_k \,&\equiv\, \iota_{v_{ij}}\dd y_k \,=\, R^{-1}\left( y_i \delta_{jk} -  y_j\delta_{ik}\right)\ , \\[1mm]
							\mathcal{L}_{v_{ij}} \mathrm{d} y_k \, &=\,  R^{-1}\left(\mathrm{d}y_i \delta_{jk} - \mathrm{d}y_j\delta_{ik}\right) \ .
						\end{split}
						\end{equation}
				The $(d-1)$-form
					%
						\begin{equation}
							\kappa_i = - \rg{*}(R\,\dd y_i) = \frac{R^{d-1}}{(d-1)!}\,\epsilon_{ij_1\ldots j_d}\, y^{j_1}\dd y^{j_2}\wedge \cdots \wedge \dd y^{j_d}
						\end{equation}
					%
				transforms under $\mathcal{L}_{v_{ij}}$ exactly as $\dd y_k$ (since $\mathcal{L}_{v_{ij}}$ preserves the round metric~\eqref{roundSdmetric}, it commutes with the Hodge star):
					%
						\begin{equation}
							\mathcal{L}_{v_{ij}} \kappa_k \,=\, R^{-1}\left(\kappa_i \delta_{jk} - \kappa_j\delta_{ik}\right) \ .
						\end{equation}
					%

				We also introduce the forms
					%
						\begin{equation}
							\begin{split}
								\omega_{ij} &= R^2\, \dd y_i \wedge \dd y_j \ , \\[1mm]
								\rho_{ij} &= \rg{*}\omega_{ij} = \frac{R^{d-2}}{(d-2)!} \,\epsilon_{ijk_1\ldots k_{d-1}} y^{k_1} \dd y^{k_2} \wedge \cdots \wedge \dd y^{k_{d-1}}\ , \\[1mm]
								\tau_{ij} &= R\left(y_i\mathrm{d}y_j  - y_j\mathrm{d}y_i\right) \otimes \rg{\rm vol}_d\ ,
							\end{split}
						\end{equation}
					%
				which transform in the adjoint representation of $\SO(d+1)$ under $\mathcal{L}_{v_{ij}}$. Namely,
					%
						\begin{equation}\label{Lie_on_omega}
							\mathcal{L}_{v_{ij}} \omega_{kl} \, = \, R^{-1}\left(\delta_{ik}\omega_{lj} - \delta_{il}\omega_{kj} - \delta_{jk}\omega_{li} + \delta_{jl} \omega_{ki} \right) \ ,
						\end{equation}
					%
				and similarly for the others, with the same overall factor $R^{-1}$.

				Furthermore, one can show the relations
					%
						\begin{align}
								\iota_{v_{ij}} \rg{\vol}_d &= \frac{R}{d-1} \mathrm{d}\rho_{ij}\, , \label{eq:contrvol}\\[1mm]
								\dd\kappa_i &= \frac{d}{R}\, y_i \rg{\rm vol}_d\ , \\[1mm]
								\dd \,\iota_{v_{ij}}\kappa_k &= - \dd\,(y_k \rho_{ij}) \ ,
						\end{align}
					%
				which are proven by making use of the trivial identity $y_{[i_1}\epsilon_{i_2\ldots i_{d+2}]}=0$.

				When computing the norm of our generalised frames, we will need the following ``squares'' of the forms defined above:
					%
						\begin{equation}\label{contractions_sphere}
							\begin{split}
								v_{ij} \,\lrcorner\, v_{kl} &= y_iy_k \delta_{jl} - y_jy_k \delta_{il} - y_iy_l \delta_{jk} +y_jy_l \delta_{ik}\ , \\[1mm]
								\omega_{ij} \,\lrcorner\, \omega_{kl} \,=\, \rho_{ij}\,\lrcorner\,\rho_{kl} &= \delta_{ik}\delta_{jl}- \delta_{il}\delta_{jk} - (y_iy_k \delta_{jl} - y_jy_k \delta_{il} - y_iy_l \delta_{jk} +y_jy_l \delta_{ik})\ ,  \\[1mm]
								\tau_{ij}\,\lrcorner\,\tau_{kl} &= y_iy_k \delta_{jl} - y_jy_k \delta_{il} - y_iy_l \delta_{jk} +y_jy_l \delta_{ik}\ ,  \\[1mm]
								\kappa_i \,\lrcorner\, \kappa_j \,=\, R^2\, \dd y_i \,\lrcorner\, \dd y_j &=  \delta_{ij} - y_iy_j \ .
							\end{split}
						\end{equation}
					%
Here, the round metric $\rg{g}$ and its inverse are used to lower/raise the indices; for instance, $\omega_{ij} \,\lrcorner\, \omega_{kl} \equiv \frac{1}{2} \rg{g}{}^{\!mp}\rg{g}{}^{\!nq}(\omega_{ij})_{mn}(\omega_{kl})_{pq}$, and so on.
				%
			
		\section{Conventions for spinors and gamma matrices}
\label{app:conv}
%======================================================================
%======================================================================
%++++++++++++++++++++++++++++++++++++++++++++++++++++++++++++++++++++++
%++++++++++++++++++++++++++++++++++++++++++++++++++++++++++++++++++++++
%======================================================================
%======================================================================
In this appendix we collect the conventions for spinors and gamma matrices 
that are relevant for the thesis, in particular for~\cref{chapbrane}. 
%%%%%%%%%%%%%%%%%%%%%%
%%%%%%%%%%%%%%%%%%%%%%%%

\subsection{\texorpdfstring{Type IIB on $\mathrm{AdS}_5 \times M_5$}{Type IIB on AdS5 x M5}}
\label{sect:IIB_notation}

We follow the conventions in~\cite{Grana_Ntokos}. The ten-dimensional metric is
\begin{equation}
	\dd  s^2 = e^{2A}\dd  s^2_{\mathrm{AdS}_5} + \dd  s^2_{M_5} \, , 
\end{equation}
and the ten-dimensional gamma matrices $\Gamma^M$ are chosen as 
\begin{equation}
	\begin{array}{lcc}
		\Gamma^{\mu} = e^{-A} \rho^{\mu} \otimes \id_4 \otimes \sigma^3\, ,& \phantom{\mbox{with}} & \mu=0, \ldots 4\, , \\
		\Gamma^{m+ 4} = \id_4 \otimes \gamma^m \otimes \sigma^1\, , & \phantom{\mbox{with}} & m=1, \ldots 5 \, ,
	\end{array}
\end{equation}
where $\rho^\mu$ and $\gamma^m$ generate $\mathrm{Cliff}(1,4)$ and $\mathrm{Cliff}(5)$ respectively, satisfying 
\begin{equation}
	\begin{array}{ccc}
		\{\rho^{\mu},\rho^{\nu}\}= 2 g^{\mu\nu}\, , & \phantom{\mbox{with}}& \{\gamma^{m},\gamma^{n}\}= 2 g^{mn} \, , 
	\end{array}
\end{equation}
and $\eta^{\mu \nu} = \mathrm{diag}(-1, 1,1,1,1)$. We also have 
\begin{equation}
	\begin{array}{ccc}
		\rho^{01 \ldots 4}=-\IIm \id\, , & \phantom{\mbox{with}}& \gamma^{1 \ldots 5}=\id\, .
	\end{array}
\end{equation}
We choose the $\mathrm{Cliff}(1,4)$ and $\mathrm{Cliff}(5)$ intertwiners as
\begin{equation}
\label{Achoice}
	\begin{array}{lcl}
		A_{1,4} = \rho_0 & \quad & C_{1,4} = D_{1,4} A_{1,4}\, , \\
		A_5 = 1 & \quad & C_5 = D_5 \, ,
	\end{array}
\end{equation}
where $C = - C^T$ in any dimension, so that 
\begin{equation}
\label{IIbintertw}
	\begin{array}{lcl}
		\rho^{\mu \, \dagger} = - A_{1,4}\rho^{\mu} A_{1,4}^{-1} & \quad \qquad \qquad & \gamma^{m \, \dagger} = \gamma^m\, , \\ 
		\rho^{\mu \, T}= C_{1,4}\rho^{\mu}C_{1,4}^{-1} & \quad \qquad \qquad & \gamma^{m \, T}= C_5 \gamma^m C_5^{-1}\, , \\ 
		\rho^{\mu *}= - D_{1,4} \rho^{\mu} D_{1,4}^{-1} & \quad \qquad \qquad & \gamma^{m *} = C_5 \gamma^m C_5^{-1} \, .
	\end{array}
\end{equation}

An explicit choice of a basis for the $\mathrm{Cliff}(1,4)$ gamma matrices is
\begin{equation}
\label{ads5gamma}
	\begin{array}{ccccc}
		\rho^0=\IIm \sigma^2 \otimes \sigma^0\, , & \rho^i = \sigma^1 \otimes \sigma^i\, , & \rho^4=-\sigma^3 \otimes \sigma^0\, , &\phantom{\mbox{and}} & i=1,2,3 \, ,
	\end{array}
\end{equation}
while for the $\mathrm{Cliff}(5)$ gamma matrices we take 
\begin{equation}
	\begin{array}{ccccc}
		\gamma^1=\sigma^1 \otimes \sigma^0\, , & \gamma^2=\sigma^2 \otimes \sigma^0\, , & \gamma^3=\sigma^3 \otimes \sigma^1\, , & \gamma^4=\sigma^3 \otimes \sigma^2\, , & \gamma^5=-\sigma^3 \otimes \sigma^3\, , 
	\end{array}
\end{equation}
with intertwiners 
\begin{equation*}
	\begin{array}{ccc}
		A_{1,4}=\rho^0\, , & C_{1,4}=\rho^0 \rho^2\, ,& C_5=\sigma^1 \otimes \sigma^2\, .
	\end{array}
\end{equation*}


With these choices for the gamma matrices the ten-dimensional chiral gamma decomposes as 
\begin{equation}
	\Gamma_{11} = \Gamma_{0, \ldots 9} = \id_4 \otimes \id_4 \otimes \sigma^2 \, . 
\end{equation}
The ten-dimensional supersymmetry parameters are Majorana-Weyl spinors of negative chirality $\Gamma_{11} \varepsilon_i = - \varepsilon_i$ ($i=1,2$) and decompose as
\begin{equation}
	\varepsilon_i = \psi \otimes \chi_i \otimes u + \psi^c \otimes \chi_i^c \otimes u\, ,
\end{equation}
%
where $\psi$ is an external $\mathrm{Spin}(4,1)$ spinor, $\chi_i$ are internal $\mathrm{Spin}(5)$ spinors and $u$ a two-component spinor satisfying
\begin{equation}
	\sigma^2 u = - u \qquad \qquad u^* = \sigma^1 u \, . 
\end{equation}

Charge conjugation of the external and internal spinors is defined as
%
\begin{align}
	\psi^c= D_{1,4} \psi^* && \chi^c= C_5 \chi^* \, . 
\end{align}
%
One can easily check that, with the above choices, 
%
\begin{equation}
\label{eq:IIB_spinor_cc_properties}
	\begin{array}{lcl}
		\psi^{cc}=-\psi\, , & \quad \quad & \left( \rho^{\mu_1} \ldots \rho^{\mu_k}\psi\right)^c= \left( -1\right)^k \rho^{\mu_1} \ldots \rho^{\mu_k} \psi^c \, , \\
		\chi^{cc}=-\chi\, , & \quad \quad & \left( \gamma^{m_1} \ldots \gamma^{m_k} \chi\right)^c= \gamma^{m_1} \ldots \gamma^{m_k} \chi^c\, .
	\end{array}
\end{equation}
%

For $5$-dimensional internal spinors, from the properties listed above, one can derive
%
\begin{equation}
	\left( \overline{\chi^c} \gamma_{m_1 \ldots m_r} \phi^c\right) = \left( \overline{\chi}\gamma_{m_1 \ldots m_r} \phi\right)^*\, ,
\end{equation}
%
and
%
\begin{equation}
	\left( \overline{\chi^c}\gamma_{m_1 \ldots m_r}\phi\right) = -\left( \overline{\chi} \gamma_{m_1 \ldots m_r} \phi^c\right)^*\, .
\end{equation}

Similarly we can derive some useful identities for the internal spinors.
Let us consider the expression,
%
\begin{equation}
	\left( \overline{\psi^c}\rho_{\mu_1 \ldots \mu_q} \psi^c\right) = \left( -1\right)^{q+1} \left( \overline{\psi} \rho_{\mu_1} \ldots \rho_{\mu_q} \psi\right)^*\, .
\end{equation}
%
where, as always, $\overline{\psi}=\psi^{\dagger}\rho_0$. Next, we obtain
%
\begin{equation}
	\left( \overline{\psi}\rho_{\mu_1} \ldots \rho_{\mu_q}\psi\right)^* = - \left( -1\right)^{\frac{q(q+1)}{2}} \left( \overline{\psi} \rho_{\mu_1} \ldots \rho_{\mu_q}\psi\right)\, .
\end{equation}
%
Combining these two equations yields
%
\begin{equation}
	\left( \overline{\psi^c}\rho_{\mu_1 \ldots \mu_q} \psi^c\right) = -\left( -1\right)^{\frac{(q+1)(q+2)}{2}} \left( \overline{\psi} \rho_{\mu_1} \ldots \rho_{\mu_q}\psi\right)\, .
\end{equation}
%
For the other combinations, one obtains
%
\begin{equation}
	\left( \overline{\psi^c}\rho_{\mu_1 \ldots \mu_q}\psi\right)^* = - \left( -1\right)^{q+1} \left( \overline{\psi} \rho_{\mu_1} \ldots \rho_{\mu_q} \psi^c\right)\, ,
\end{equation}
%
and 
%
\begin{equation}
	\left( \overline{\psi^c} \rho_{\mu_1 \ldots \mu_q}\psi\right)^* = -\left( -1\right)^{\frac{q(q+1)}{2}} \left( \overline{\psi} \rho_{\mu_1 \ldots \mu_q} \psi^c\right)\, . 
\end{equation}
%
Again combining the last two relations gives
%
\begin{equation}
	 \left( \overline{\psi}\rho_{\mu_1 \ldots \mu_q} \psi^c\right) = \left( -1\right)^{\frac{(q+1)(q+2)}{2}}\left( \overline{\psi}\rho_{\mu_1 \ldots \mu_q} \psi^c\right)\, ,
\end{equation}
%
so that these terms vanish for $q=0,1,4$. 



%%%%%%%%%%%%%%%%%%%%%
%========M-theory section========%
%%%%%%%%%%%%%%%%%%%%%
\subsection{M-theory}
\label{app:mtheoryconv}
%
We follow again the conventions in~\cite{Grana_Ntokos} for the metric ansatz
\begin{equation}
	\dd  s^2 = e^{2\Delta}\dd  s^2_{\mathrm{AdS}} + \dd  s^2_{M} \, .
\end{equation}
We consider two M-theory setups: $\mathrm{AdS}_4$ compactifications with a $7$-dimensional internal manifold $M_7$
and $\mathrm{AdS}_5$ ones on a $6$-dimensional internal manifold $M_6$.
The eleven-dimensional gamma matrices are $\hat{\Gamma}^M$, $M=0,\ldots, 10$, satisfying the Clifford algebra $\mathrm{Cliff}(1,10)$ relations, 
\begin{equation}
	\{\hat{\Gamma}^A , \hat{\Gamma}^B \} = 2 \eta^{AB}\, .
\end{equation}
They will decompose as in~\eqref{eq:dec_gammas_m6} and~\eqref{eq:dec_gammas_m7}, and for convenience, we report them here,
\begin{equation}
\label{11gammadec}
	\begin{array}{lccr}
		\hat{\Gamma}^\mu = e^{-\Delta}\,\rho^\mu \otimes \Gamma_7\, , & \hat{\Gamma}^{m+4} = \id_4 \otimes \Gamma^m\, & \phantom{\mbox{for}} &\mbox{for}~\mathrm{AdS}_5 \times M_6\, , \\[2mm]
		\hat{\Gamma}^\mu = e^{-\Delta}\,\rho^\mu \otimes \id_8\, , & \hat{\Gamma}^{m+3} =e^{-\Delta} \rho_5 \otimes \Gamma^m\, &\phantom{\mbox{for}} &\mbox{for}~\mathrm{AdS}_4 \times M_7\, .
	\end{array}
\end{equation} 
In the expressions above we denoted the internal gamma matrices with the same symbol for both cases, with $\Gamma_7$ the chiral operator in $6$ dimensions, defined below, and with $\rho_5$ the external $\mathrm{Cliff}(1,4)$ chiral operator.
%%%%%
%%%%%%%% 
%%%%%
\subsubsection{\texorpdfstring{M-theory on $\mathrm{AdS}_5 \times M_6$}{M-theory on AdS5 x M6}}
%
Here we give the conventions for M-theory solutions of the form $\mathrm{AdS}_5 \times M_6$. The external part is the same as the type IIB compactification.
So we refer to~\cref{sect:IIB_notation}. For the internal part we take as reference~\cite{VanProeyen:1999ni} and we write all the gamma matrices as tensor products of Pauli matrices
\begin{equation}
\label{gamma6}
	\begin{aligned}
		\Gamma^1 &= \sigma_1 \otimes \id \otimes \sigma_{1}\, , \\[1mm]
		\Gamma^2 &= \sigma_1 \otimes \id \otimes \sigma_{2}\, , \\[1mm]
		\Gamma^3 &= \sigma_1 \otimes \sigma_{1} \otimes \sigma_{3}\, , \\[1mm]
		\Gamma^4 &= \sigma_2 \otimes \sigma_{1} \otimes \sigma_{3}\, , \\[1mm]
		\Gamma^5 &= -\sigma_1 \otimes \sigma_{3} \otimes \sigma_{3}\, , \\[1mm]
		\Gamma^6 &= \sigma_3 \otimes \id \otimes \id\, . 
	\end{aligned}
\end{equation}
In even dimensions we can define a chiral operator
\begin{equation}
\label{chiralgamma}
	\Gamma_7 = - \IIm \Gamma^1 \ldots \Gamma^6\, .
\end{equation}
The chiral operator $\Gamma_7$ squares to the identity and satisfies the following relations with the other gamma matrices
\begin{align*}
	\left\{\Gamma_7\, , \Gamma_{m} \right\} &= 0\, ,\\
	\left[\Gamma_7\, , \Gamma_{mn} \right] &= 0\, .
\end{align*}
By induction, these properties can be extended to any odd/even rank element of the Clifford algebra.

The intertwiners of $\mathrm{Cliff(6)}$ can be written as follows
\begin{align*}
\Gamma_m^T & = C_6^{-1} \Gamma_m C_6 \, , \\
\Gamma_m^* & = D_6^{-1} \Gamma_m D_6 \, , \\
\Gamma_m^{\dagger} & = A_6 \Gamma_m A^{-1}_6 \, .
\end{align*}
and, for our conventions, 
\begin{equation*}
	\begin{array}{lcr}
		A_6 = 1 \, ,& & D_6 = C_6\, .
	\end{array}
\end{equation*}


%======================================
%%%%%%%%%%%%%%%%%%%%%%%%%%
%======================================
%%%%%%%%%%%%%%%%%%%%%%%%%%
\subsubsection{\texorpdfstring{M-theory on $\mathrm{AdS}_4 \times M_7$}{M-theory on on AdS4 x M7}}
Here we give the conventions which are relevant for the $\mathrm{AdS}_4$ solutions of M-theory. We made the choice of having compatible conventions with the previous section, such that one can embed all the relations above in the following ones.

The Clifford algebra $\mathrm{Cliff}(7)$ and its generators are constructed by the same set of gamma matrices~\eqref{gamma6} of $\mathrm{Cliff}(6)$ 
plus the chiral gamma $\Gamma_7$ in~\eqref{chiralgamma}.


The $\mathrm{Cliff}(7)$ intertwiners are written as
\begin{align*}
	\Gamma_m^T &= C_7^{-1} \Gamma_m C_{7}\, , \\[1mm]
	\Gamma_m^\dagger &= A_7 \Gamma_m A_7^{-1}\, , \\[1mm]
	\Gamma_m^* &= D_7^{-1} \Gamma_m D_7\, .
\end{align*}
%
Numerically the matrices $A_7$, $C_7$, $D_7$ are the same as $A_6$, $C_6$, $D_6$.

The four-dimensional gamma matrices on $\mathrm{AdS}_4$ satisfy 
\begin{equation}
	\{\rho_a , \rho_b \} = 2 \eta_{ab}\id \, ,
\end{equation}
where $a, b$ are frame indices. Hence it holds $\eta^{ab}e^\mu_a \otimes e^\nu_b = g^{\mu\nu}$, where $g^{\mu\nu}$ is the $\mathrm{AdS}_4$ inverse metric. 
In terms of flat frame indices, we choose a basis for explicit calculations for $\mathrm{Cliff}(1,3)$,
\begin{equation}
\begin{array}{lrcc}
	\rho^0=\IIm \sigma^2 \otimes \sigma^0\, , & \rho^i = \sigma^1 \otimes \sigma^i\, , &\phantom{\mbox{and}} & i=1,2,3 \, . 
\end{array}
\end{equation}
As for the internal part we have chosen the basis above such that we can embed it into~\eqref{ads5gamma}.
The intertwiners can be written as in~\eqref{Achoice},
\begin{equation}
	\begin{array}{ccc}
		\rho^{\mu\dagger} = -A_{1,3} \rho^\mu A_{1,3}^{-1}\, , &\phantom{\mbox{for}} &A_{1,3} = \rho_0\, , \\[1mm]
		\rho^{\mu T} = C_{1,3} \rho^\mu C_{1,3}^{-1}\, , &\phantom{\mbox{for}} & C_{1,3} = D_{1,3} A_{1,3} \, , \\[1mm]
		\rho^{\mu *} = -D_{1,3} \rho^\mu D_{1,3}^{-1}\, . & \phantom{\mbox{for}} & 
	\end{array}
\end{equation}

		%
\end{document}